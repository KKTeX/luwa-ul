\documentclass[luatex,fontsize=8pt,paper=b5,twoside]{jlreq}%
\usepackage{luwa-ul}
\usepackage{hyperref}
\usepackage{listings}
\usepackage{caption}
\usepackage[most]{tcolorbox}
\usepackage{fp}
\usepackage{lltjext}
\usepackage{luatexja-ruby}

\lstset{
    basicstyle=\ttfamily\small,
    keywordstyle=\color{blue},
    commentstyle=\color{gray},
    stringstyle=\color{red},
    breaklines=true,
    breakatwhitespace=false,  
    columns=flexible           
}

\usepackage{hyperref} 
\hypersetup{
  luatex, pdfencoding=auto, 
  colorlinks=true,
  linkcolor=black,     
  citecolor=black,     
  urlcolor=DeepSkyBlue3,      
  pdfborder={0 0 0}, 
}

\colorlet{grayLight}{white!80!black} 

\NewTCBListing{SourceCode}{ O{Input} m !o !O{DeepSkyBlue3} }{%
  enhanced, colback=black!70, colframe=Snow4,
  toptitle=-1mm, bottomtitle=-1mm,
  righttitle=-1mm, lefttitle=-1mm,
  arc=.5mm, 
  title={\tcbox[on line, arc=.5mm, boxsep=0pt, boxrule=0pt, top=1mm, bottom=0.8mm, left=2mm, right=2.2mm, colback=gray!80, coltext=white]{\raisebox{-.1ex}{\vphantom{羅}#1}}},fonttitle=\gtfamily\footnotesize\bfseries,boxrule=0.8pt,
  breakable,before upper={\color{white}},top=-0.5mm,bottom=-0.5mm,
  after title=\IfNoValueTF{#3}{}{{\hfill\tcbox[on line, arc=.5mm, boxsep=0pt, boxrule=0pt, top=1mm, bottom=0.8mm, left=2mm, right=2.2mm, colback=grayLight, coltext=#4]{\raisebox{-0.1ex}{\vphantom{羅}#3}}}},
  listing only,
  listing options={
    language={#2},
    basicstyle=\ttfamily,
    keywordstyle=\ttfamily\color{white},
    stringstyle=\itshape\color{white},
    commentstyle=\small\gtfamily\color{DeepSkyBlue2},
    showspaces=false,showtabs=false,
    breaklines=true,breakindent=0pt,
    showstringspaces=false,
    columns=fullflexible,
    tabsize=2,
    numbers=left,numbersep=1.5pt,
    numberstyle=\scriptsize\gtfamily\color{gray},
  }
}

\NewTColorBox{OutPut}{ O{Output} !o !O{DeepSkyBlue3} }{%
  enhanced, colframe=Snow4,
  toptitle=-1mm, bottomtitle=-1mm,
  righttitle=-1mm, lefttitle=-1mm,
  arc=.5mm, colback=white, 
  title={\tcbox[on line, arc=.5mm, boxsep=0pt, boxrule=0pt, top=1mm, bottom=0.8mm, left=2mm, right=2.2mm, colback=gray!40, coltext=DeepSkyBlue3]{\raisebox{-.1ex}{\vphantom{羅}#1}}},fonttitle=\gtfamily\footnotesize\bfseries,boxrule=0.8pt,
  breakable,top=-0.5mm,bottom=-0.5mm,
  after title=\IfNoValueTF{#2}{}{{\hfill\tcbox[on line, arc=.5mm, boxsep=0pt, boxrule=0pt, top=1mm, bottom=0.8mm, left=2mm, right=2.2mm, colback=grayLight, coltext=#3]{\raisebox{-0.1ex}{\vphantom{羅}#2}}}}, bottom=2mm, top=2mm, 
}

\title{\texttt{luwa-ul} Package Documentation}
\author{Kosei Kawaguchi a.k.a. KKTeX}
\date{Version 1.0.0 (2025/10/23)}

\begin{document}
\begin{titlepage}
  \maketitle
\end{titlepage}
\newpage
\tableofcontents
\newpage

\section{設置(Installation)}
インストール方法は、適切な箇所に\texttt{luwa-ul.sty}を配置し、\verb|\usepackage{luwa-ul}|と書き込むだけで完了です。オプションはありません。
(Place \texttt{luwa-ul.sty} in a directory where LaTeX can find it.)

依存性についてです。(Dependencies: )

\begin{itemize}
  \item luacolor, xcolor
  \item lua-ul
  \item calc, tikz
\end{itemize}

\noindent を内部的に仕様しています。(This package uses them internally.)

\section{注意(Caution)}
このパッケージはその実装の上でLua言語を仕様しているので、LuaLaTeXの上でのみ仕様できるパッケージとなっています。(This package internally uses Lua, so it can be compiled only by LuaLaTeX.)

\section{提供されるコマンド}
提供されるコマンドは以下の通りです。要望に応じて追加されることもありえます。
(The provided commands are as follows. I would add some more commands upon requests.)

\begin{itemize}
  \item \verb|\underLineKK|, \verb|\overLineKK|
  \item \verb|\nijyusen|, \verb|\keshinijyusen|
  \item \verb|\thinHighLight|, \verb|\thickHighLight|
  \item \verb|\stength|, \verb|\sdashgth|, \verb|\snamift|
\end{itemize}

\subsection{\textbackslash underLineKK}
このコマンドは、\texttt{lua-ul}パッケージによって提供される\verb|\underLine|コマンドに対し、

\begin{itemize}
  \item 縦書き時のずれの解消
  \item ルビがあった際の挙動の変化
\end{itemize}

\noindent の2点において調整を施したものとなっています。

(This command is basically made out of \verb|\underLine| command, which is provided by \texttt{lua-ul} package. I added some adjustments on it in order to 
\begin{itemize}
  \item eliminate the misalignment in vertical writing mode.
  \item change its behavior when the argument includes phonetic guides.
\end{itemize}
)

使い方は以下のようになります。

\begin{SourceCode}{TeX}
  \underLineKK[color=色, bottom=下線最下部の位置のずれ, heigth=下線の太さ(bottomの位置からの距離で計算)]{引数}

  \underLineKK[color=red]{あああああああああ}
  \underLineKK[bottom=.1em]{あああああああああ}

  \parbox<t>[c][6cm][c]{3cm}{%
    \underLineKK{ああ\ruby{あ}{あ}あああああ}
  }
\end{SourceCode}

\begin{OutPut}
  横書き:

  \underLineKK[color=red]{ああああ\footnote{あ}{ああああ}あああああ}
  \underLineKK[bottom=-.5em]{あああああああああ}

  縦書き:

  \parbox<t>[c][6cm][c]{3cm}{%
    \underLineKK{ああ\ruby{あ}{あ}あああああ}
     \underLineKK{ああああああ}
  }
\end{OutPut}

\end{document}